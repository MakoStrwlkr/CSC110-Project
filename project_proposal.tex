\documentclass[fontsize=11pt]{article}
\usepackage{amsmath}
\usepackage[utf8]{inputenc}
\usepackage[margin=0.75in]{geometry}
\usepackage{makecell}

\title{CSC110 Project1: Project Phase 1: Proposal}
\author{Hanrui Fan, Ching Chang, Arkaprava Choudhury, Letian Cheng}
\date{\today}

\begin{document}
\maketitle

\textbf{\large 1. Project title (pick something informative and professional, but you can be creative too) and name of all group members.}

\textbf{\large 2. Brief problem description and research question. (300–400 words)}

\begin{itemize}
  \item Give an overview of any background knowledge necessary for the reader to understand the problem you are studying.
  \item Provide context for the problem and motivate why you have chosen your research question.
  \item Your research question should be in \emph{bold}; it should be fairly concise, but can be more than one sentence.
\end{itemize}


\textbf{\large 3. A description of at least one relevant dataset you have found. (~150 words)}

\begin{itemize}
    \item State the source (e.g., government/organization website) and format (e.g., text, csv, json, image) of the dataset, and give some sample data contained inside that dataset.
    \item Don’t be afraid to cobble together your own dataset, such as creating a collection of images that are related. Or to combine two datasets from different sources.
    \item You will also submit a small sample of your dataset to MarkUs along with your project proposal document. (See more below)
\end{itemize}

The data we show here is based on joint estimates by Brazilian National Institute of Space Research and the United Nations Food and Agriculture Organization with map data provided by MapBiomas.\\
    The data is presented as a text-based table, but we will convert it into a csv file. The data shows the area of the Amazon rainforest in Brazil, which accounts for about 60% of the total area of the country.\\
    For these data we did some pre-processing, we first removed items in the data that had NA (unrecorded or incomplete data), these included data up to 1985 and 2019, but we needed to record data from pre-1970 as this is the baseline data for the subsequent percentage change.\\
    We then selected data from the start of 1988 because our team considered the absence of the change in deforestation loss data between 1978-1987 to be unrepresentative, so we filtered the 1978-1987 data as well.\\
    At the same time, these data contain four columns including estimated remaining forest cover, average annual forest area loss, percentage of forest cover compared to 1970 area, and total forest area lost since 1970.\\
    
\textbf{\large 4. A computational plan for your project. (300–500 words)}

\begin{itemize}
    \item Describe the kinds of computations you plan to perform, such as: data transformation/filtering/aggregation, computational models, and/or algorithms.
    \item Explain how your program will report the results of your computation in a visual and/or interactive way. You don’t need to go into a lot of details here, but it should be clear what you plan to do.
\end{itemize}

\textbf{Technical requirement:} for your project, you must use at least one Python library/module that we have not covered in this course, \emph{or} use \underline{plotly} or \underline{pygame} to a much larger extent than what what have given you so far in this course. (See examples and note in the next section).

\begin{itemize}
    \item In this part of your proposal, you should also describe one new library you intend to use, how you will use it, and why it is appropriate. Refer to specific functions, data types, and/or capabilities of the library that make it relevant for solving the problem you wish to solve.
\end{itemize}

\textbf{\large 5. A references section that lists the references you used for your proposal. This should include references from your topic research, the reference for where you obtained the dataset, and any online documentation or tutorials for the Python library you plan to use for the project.}

\begin{itemize}
    \item You may use any academic reference style you wish, e.g. APA or MLA.
\end{itemize}

\end{document}
