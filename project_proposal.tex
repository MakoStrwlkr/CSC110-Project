\documentclass[12pt]{article}
\usepackage[margin=1in]{geometry}
\usepackage{amsmath, amssymb, amsthm}
\usepackage{tcolorbox}
\usepackage{lastpage}
\usepackage{fancyhdr, accents}
\usepackage{natbib}

\pagestyle{fancy}
\setlength{\headheight}{40pt}

\newcommand{\ubar}[1]{\underaccent{\bar}{#1}}

\newcommand\tab[1][1cm]{\hspace*{#1}}

\title{CSC110Y1-F Fall 2020 - Fundamentals of Computer Science 1 \\ Course Project Proposal}
\author{
  Ching Chang\\
  \and
  Letian Cheng\\
  \and
  Arkaprava Choudhury\\
  \and
  Hanrui Fan
}
\date{\today}

\begin{document}

\maketitle
% Make cover page. Not to be submitted, but for aesthetic.

\newpage

\lhead{Ching Chang, Letian Cheng \\ Arkaprava Choudhury, Hanrui Fan}
\rhead{CSC110Y1-F Fall 2020 \\ Fundamentals of Computer Science 1 \\ Course Project Proposal}
\cfoot{\thepage\ of \pageref{LastPage}}

\begin{enumerate}
\item \section*{Part 1}

\newpage

\item \section*{Part 2}

\begin{text}
Research Question: \textbf{To what extent do the changes in the Amazon rainforest’s area affect the local annual precipitation and air quality?}

During our initial research of topics related to climate change, we found that the South American rainforest contributes 20\% \citep{Tho20} of the oxygen produced by photosynthesis on land,
while the Amazon rainforest is responsible for 10\% of the current greenhouse gas emissions \citep{Mel93}.
This information was surprising to us since 20\% of photosynthesis implies a lot of conversion from carbon dioxide,
which is a greenhouse gas, to oxygen, whereas the 10\% contribution to greenhouse gas seems to contradict that information.
Due to this contradiction, we were curious about whether tree populations actually help control the climate.
After some research, we learned that the effect of trees on climate change is more complex than we originally thought.
There are many factors to consider, such as the carbon dioxide to oxygen conversion, the tendency to trap heat due to their dark color,
reaction to form methane and ozone, and political movements revolving tree plantations \citep{Mar20}.
This led us into choosing an empirical approach —we wanted to directly observe the relationship between the change of tree population and climate change.
We chose to focus our data on the Amazon rainforest not only because it is the largest rainforest on Earth \citep{WWF13},
but also because there have been several pieces of evidence that show that the Amazon rainforest has been suffering from deforestation recently.
For example, over 700,000 km2 (270,000 mi2) of Amazon rainforest had been lost since 1970, reducing its size to 80.7\% of its original size, in 2018 \citep{But20};
There have been more than 40,000 fires in the rainforest in 2019 \citep{Gov20}; and that forest exploitation in Amazon has risen for 14 consecutive months in June 2020 \citep{Reu20}.
With these major evidence of deforestation correlating to the change in global and local climate, we believe that it is a relevant topic to contemporary society that should not be ignored.
\end{text}

\newpage

\item \section*{Part 3}

\begin{text}

\end{text}

\newpage

\item \section*{Part 4}

\begin{text}

We first create a function that parses the \texttt{html} element of the stats on the website as a string, and converts it to a \texttt{nested array} so that it’s easier to work with. This will involve using a \texttt{for loop}, \texttt{if statements}, and an \texttt{accumulator} keeping track of the data parsed so far. Using this function, we will collect our data for deforestation in the Amazon rain-forest over the past few decades. With the data now converted into a form that we can easily manipulate, we shall focus on analysing the data using our own functions.

For this project, we use smooth polynomial fitting to relate two of the variables in our \texttt{nested list}. Now, although there exist readily available functions that would do the same in the module scipy, we try implementing our own functions for the same, to test our learning from the course.

We split the mathematical algorithm for this problem using top-down design. Firstly, the main function would have two lists, \texttt{l\_x, l\_y}, of same length as input (for the two variables), along with an integer $n$ (where $n \leq \texttt{len(l\_x)}$; representing degree of intended polynomial). The function body would have calls to helper functions. Note: this is only a rough outline and the exact technical details may be changed based on the results after testing the functions.

Firstly, we have a function to calculate the perpendicular distance of one point from a given polynomial. As opposed to the naive approach to the problem, we use Newton-Raphson method \textit{repeatedly} to estimate a solution for the derivative of the expression for the difference between the point and the polynomial, hence finding the coordinates of the foot of perpendicular, and consequently, the length of perpendicular.

The first estimate for the polynomial will be trivial, and we will then run the simplex algorithm to minimize the sum of the squares of the perpendiculars using two more helper functions, yielding the polynomial regression model. We now move towards plotting the resulting graph using the matplotlib library.

We also plan to write a function to calculate the coefficient of determination to check whether the graph shows an appropriate relationship between the independent variable (forest cover) and the dependent variable.

In addition to the graph, we plan to create an interactive text-based report of our data, where the user  inputs a value for the independent or dependent variable, and the program will provide the corresponding dependent or independent value, coefficient of determination, or the slope of the tangent at the point, depending on which one the user asks for. The output will be text-based, and will require string concatenation, and if statements to check whether to add trivial information to the report.

The input/output model will use while loops and input prompts to keep the program interactive. We also extrapolate the data to yield the predictions about future data using the interactive i/o model. Finally, we use the extrapolated data to summarize the upcoming significant years where the dependent variable will reach a certain milestone.


\textbf{Technical requirement:}

The Annual total $CO_2$ emissions dataset we have is a CSV file. In order to parse it easily, We decide to use pandas for CSV file reading.

Pandas is a python library written for data manipulation and analysis.
In particular, it offers data structures and operations for manipulating numerical tables and time series.

Specific, we will use the function "read\_csv". This function have many parameters available for us, but we will only use "filepath\_or\_buffer" parameter.
It takes a valid string path, which is the path to our CSV file.

This function will return a class - Pandas DataFrame that contains the data of the CSV file. Pandas DataFrame is easier for calculating dataset thanks to its various build in functions.
In this way, we can access each cell of CSV file for our later research.

\end{solution}

\maketitle

\newpage

\bibliography{references}
\bibliographystyle{apalike}

\end{enumerate}

\end{document}